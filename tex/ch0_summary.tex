Asteroseismology allows us to probe the internal structure of stars through their global modes of oscillation. 
Thanks to missions such as the NASA \emph{Kepler} space observatory, we now have high-quality asteroseismic data for nearly $100$ solar-type stars. 
This presents an opportunity to measure the core structures of these stars as well as their ages, masses, radii, and other fundamental parameters. 

This thesis is primarily concerned with two inverse problems in asteroseismology. 
The first is to estimate the fundamental parameters of stars from observations using evolutionary arguments. 
This is inverse to the forward problem of simulating the theoretical evolution of a star, given the initial conditions. 
We solve this problem using supervised machine learning in Chapter~\ref{chap:ML}. 
We find ages, masses, and radii of stars with uncertainties (in the sense of precision) better than $6\%$, $2\%$, and $1\%$, respectively. 
We furthermore use unsupervised machine learning to quantify how each kind of observation of a star is related to its fundamental parameters in Chapter~\ref{chap:statistical}. 

The second problem is to infer the structure of a star from its frequencies of pulsation using asteroseismic arguments. 
This is inverse to the forward problem of calculating the theoretical pulsation frequencies for a known stellar structure. 
Solving this problem presents an opportunity to test the quality of stellar evolution models, as we may then directly compare the asteroseismic structure of a star against theoretical predictions. 
We solve this problem in Chapter~\ref{chap:inversion}. 
Applying this technique to the solar-type stars in 16~Cygni, we find that while the structure of the $1.03$ solar-mass star 16~Cyg~B is in good agreement with theoretical expectations, the more massive 16~Cyg~A differs in its internal structure from best-fitting evolutionary models. 

These inverse problems are both \emph{ill-posed} in the sense that (I) a solution may not exist within the confines of the current theory; (II) if there is a solution, it may not be unique, as many solutions may be consistent with the data; and/or (III) the solutions may be unstable with respect to small fluctuations in the input data. 
Therefore, care must be put into determining possible solutions and applying regularization where necessary. 

Chapter~\ref{chap:intro} introduces this thesis with the history and theory of stellar structure, evolution, and pulsation; and emphasizes the role that variable star astronomy played in shaping our understanding of stellar evolution. 
It also contains the kernels of stellar structure, an introduction to ill-posed inverse problems, and a discussion of some computational issues for the algorithms used to solve these problems. 
